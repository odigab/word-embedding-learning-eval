%
% File acl2013.tex
%
% Contact  navigli@di.uniroma1.it
%%
%% Based on the style files for ACL-2012, which were, in turn,
%% based on the style files for ACL-2011, which were, in turn, 
%% based on the style files for ACL-2010, which were, in turn, 
%% based on the style files for ACL-IJCNLP-2009, which were, in turn,
%% based on the style files for EACL-2009 and IJCNLP-2008...

%% Based on the style files for EACL 2006 by 
%%e.agirre@ehu.es or Sergi.Balari@uab.es
%% and that of ACL 08 by Joakim Nivre and Noah Smith

\documentclass[11pt]{article}
\usepackage{acl2013}
\usepackage{times}
\usepackage{url}
\usepackage{latexsym}
\usepackage{graphicx}
\usepackage{color}
\usepackage[skip=0pt]{caption}
\usepackage{subcaption}
\usepackage{subfig}
\usepackage{pgfplotstable}
\usepackage{adjustbox}
\usepackage{booktabs}
\usepackage{amssymb}
\usepackage{amsmath}
\usepackage{enumerate}
\usepackage{paralist}
\usepackage{xspace}


\newcommand{\RQ}[1][1]{\textbf{RQ#1}\xspace}

\newcommand{\figref}[2][]{Fig#1.~\ref{#2}\xspace}
\newcommand{\tabref}[2][]{Tab#1.~\ref{#2}\xspace}

\newcommand{\lex}[1]{\textit{#1}\xspace}

\newcommand{\dataset}[1]{\texttt{#1}\xspace}
\newcommand{\EWT}{\dataset{EWT}}
\newcommand{\WSJ}{\dataset{WSJ}}
\newcommand{\Brown}{\dataset{Brown}}
\newcommand{\Reuters}{\dataset{Reuters}}
\newcommand{\MUC}{\dataset{MUC7}}

\newcommand{\method}[2][]{\ensuremath{\textsc{#2#1}}\xspace}
\newcommand{\unigram}[1][]{\method{Unigram}}
\newcommand{\brown}[1][]{\method[\ensuremath{_{#1}}]{Brown}}
\newcommand{\CW}[1][]{\method[#1]{CW}}
\newcommand{\CBOW}[1][]{\method[#1]{CBOW}}
\newcommand{\Skipgram}[1][]{\method[#1]{Skip-gram}}
\newcommand{\Glove}[1][]{\method[#1]{Glove}}
\newcommand{\withup}{\method{+UP}}

\newcommand{\task}[1]{\textsf{#1}\xspace}
\newcommand{\pos}{\task{POS-tagging}}
\newcommand{\chunking}{\task{Chunking}}
\newcommand{\ner}{\task{NER}}
\newcommand{\mwe}{\task{MWE}}

\newcommand{\evmeasure}[1]{\textsc{#1}\xspace}
\newcommand{\accuracy}{\evmeasure{Acc}}
\newcommand{\fscore}{\evmeasure{F1}}


\newcommand{\best}[1]{\textbf{#1}}

\newcommand{\ctx}{\ensuremath{\text{ctx}}}


\hyphenation{an-aly-sis}
\hyphenation{an-aly-ses}
\hyphenation{an-aly-ser}

\title{Big Data Small Data, In Domain Out-of Domain, Known Word Unknown
  Word: The Impact of Word Representation on Sequence Labelling Tasks}

\author{A Anonymous 
   \\%NICTA / Locked Bag 8001, \\ Canberra ACT 2601, Australia \\
   \\ %The Australian National University\\
   \\ %University of Canberra \\
  \\ % {\tt \small{@nicta.com.au}} \\
\And
  B Anonymous
   \\%NICTA / Locked Bag 8001, \\ Canberra ACT 2601, Australia \\
   \\%The Australian National University\\ \\
   \\ %{\tt \small{@nicta.com.au}} \\
}

\date{}

% Max 8 pp.

\begin{document}


\maketitle


\begin{abstract} 
  Word embeddings -- distributed word representations that can be
  learned from unlabelled data -- have been shown to have high utility
  in many natural language processing applications. 
  In this paper, we perform an extrinsic evaluation of four popular word
  embedding methods in the context of four sequence labelling tasks:
  POS-tagging, syntactic chunking, NER and MWE identification.
  A particular focus of the paper is analysing the effects of task-based
  updating of word representations.
  We show that when using word embeddings as features, as few as
  several hundred training instances are sufficient to achieve competitive
  results, and that word embeddings lead to improvements over OOV words
  and out of domain.
  Perhaps more surprisingly, our results indicate there is little
  difference between the different word embedding methods, and that simple
  Brown clusters are often competitive with word embeddings across all
  tasks we consider. 
\end{abstract}

\newcommand{\gabi}[1]{\textcolor{blue}{#1}}
\newcommand{\tim}[1]{\textcolor{red}{#1}}
\newcommand{\lizhen}[1]{\textcolor{green}{#1}}
\newcommand{\nss}[1]{\textcolor{magenta}{#1}}

\section{Introduction}

Recently, distributed word representations have grown to become a
mainstay of natural language processing (NLP), and been show to have
empirical utility in a myriad of tasks
\cite{Collobert2008,turian2010word,baroni:2014,Andreas:Klein:2014}.  The
underlying idea behind distributed word representations is simple: to
map each word $w$ in our vocabulary $V$ onto a continuous-valued vector
of dimensionality $d \ll |V|$.  Words that are similar
(e.g., with respect to syntax or lexical semantics) will ideally be mapped to
similar regions of the vector space, implicitly supporting both
generalisation across in-vocabulary (IV) items, and countering the
effects of data sparsity for low-frequency and out-of-vocabulary (OOV)
items.

Without some means of automatically deriving the vector representations
without reliance on labelled data, however, word embeddings would have
little practical utility. Fortunately, it has been shown that they can
be ``pre-trained'' from unlabelled text data using various algorithms 
to model the distributional hypothesis (i.e., that
words which occur in similar contexts tend to be semantically
similar). Pre-training methods have been refined considerably in recent
years, and scaled up to increasingly large corpora.

As with other machine learning methods, it is well known that the
quality of the pre-trained word embeddings depends heavily on factors
including parameter optimisation, the size of the training data, and the
fit with the target application. For example, \newcite{turian2010word}
showed that the optimal dimensionality for word embeddings is task-specific.  
One factor which has received relatively little attention in
NLP is the effect of ``updating'' the pre-trained word embeddings as
part of the task-specific training, based on self-taught
learning~\cite{raina2007self}.  Updating leads to word
representations that are task-specific, but often at the cost of
over-fitting low-frequency and OOV words.


In this paper, we perform an extensive evaluation of five word embedding
approaches under fixed experimental conditions, applied to four sequence
labelling tasks: POS-tagging, full-text chunking, named entity
recognition (NER), and multiword expression (MWE) identification. In
this, we explore the following research questions:
\begin{compactenum}[\bf RQ1:]
\item are word embeddings better than baseline approaches of one-hot
  unigram features and Brown clusters?
\item do word embeddings require less training data (i.e.\ generalise
  better) than one-hot unigram features?
\item what is the impact of updating word embeddings in sequence
  labeling tasks, both empirically over the target task and
  geometrically over the vectors?
\item what is the impact of word embeddings (with and without
  updating) on both OOV items (relative to the training data) and
  out-of-domain data?
\item overall, are some word embeddings better than others in a sequence
  labelling context?
\end{compactenum}



\section{Word Representations}
\label{wordrep}
The distributional hypothesis in linguistics suggests that ``a word is characterised by the company it keeps''~\cite{firth1957}. Words that are used in the similar contexts tend to have similar semantic and syntactic properties. Capturing distributional similarity is the underlying idea of all word representation learning methods. 

\subsection{Types of Word Representations}
Based on the ways of constructing word representations, \newcite{turian2010word} categorised these methods into three types: \textit{Distributional representation},  \textit{Cluster-based representation}, and \textit{Distributed representation}.

\textit{Distributional representation} methods map each word $w$ to its context word vector $\mathbf{C}_w$, which is built based on co-occurrence counts between $w$ and the words surrounding it. The learning methods store either directly the co-occurrence counts between two words $w$ and $i$ in $C_{wi}$~\cite{sahlgren2006word,turney2010frequency,honkela1997self} or project the concurrence counts between words into a lower dimensional space~\cite{vrehuuvrek2010software,lund1996producing} by applying dimension reduction techniques such as SVD~\cite{dumais1988using} and LDA~\cite{blei2003latent}. 

The methods of \textit{Cluster-based representation} build clusters of words by applying either soft- or hard clustering algorithms~\cite{lin2009phrase,li2005semi}. Some of them also rely on co-occurrence matrix of words~\cite{pereira1993distributional}. The Brown clustering algorithm~\cite{Brown92class-basedn-gram} is the most famous in this category.

A \textit{distributed representation} of words takes the form of a dense, low-dimensional, and continuous-valued vector. It is compact and stores mutually non-exclusive latent features. This kind of representations are also called word embeddings, which are built in the hope of capturing both syntactic and semantic properties of words.

\subsection{Selected Word Representations}
In a range of sequence tagging tasks, we evaluated five word representations : Brown clustering, Collobert \& Weston (CW)~\cite{collobert2011natural}, continuous bag-of-words model (CBOW)~\cite{Mikolov13}, continuous skip-gram model (Skip-gram)~\cite{Mikolov13NIPS}, and Global vectors (Glove)~\cite{pennington2014glove}. Except CW, all other methods are ranked as the best method in different recent empirical studies~\cite{turian2010word,pennington2014glove}. CW is included because it is one of the most influential early work in this area. The training of these word representations is conducted in an unsupervised way. The underlying idea is to predict occurrence of words in the neighbourhood. Their training objectives share the same form, which is a sum of local training factors $J(w, \text{ctx}(w))$. 
\begin{displaymath}
L = \sum_{w \in V} J(w, \text{ctx}(w))
\end{displaymath}
where $V$ is the vocabulary set of a corpus and $\text{ctx}(w)$ denotes the local context of a word $w$. The local context of a word can either be its previous $k$ words or $k$ words surrounding it. Local training factors are designed to capture the relationship between current words and their local contexts. They either predict current words based on local contexts, or use current words to estimate their context words. Except Brown clustering, which utilises cluster-based representation, all other methods employ distributed representation.

The starting point of CBOW and Skip-gram models is to employ softmax for predicting word occurrence, then
\begin{displaymath}
J(w, \text{ctx}(w)) = - \log \big ( \frac{\exp(\mathbf{v}_w^{\text{T}} \mathbf{v}_{\text{ctx}(w)})}{ \sum_{j \in V} \exp(\mathbf{v}_j^{\text{T}} \mathbf{v}_{\text{ctx}(w)})} \big )
\end{displaymath}
where $\mathbf{v}_{\text{ctx}(w)}$ denotes the distributed representation of the local context of word $w$. CBOW takes the average of the representation of all context words as $\mathbf{v}_{\text{ctx}(w)}$. Thus it estimates the probability of current words $w$ given its local context. In contrast, Skip-gram applies softmax for each context word of a word $w$. In this case, $\mathbf{v}_{\text{ctx}(w)}$ corresponds to the representation of one of its context words. This model is interpreted as predicting context words based on current words. In practice, softmax is too expensive to compute thus~\newcite{Mikolov13NIPS} propose to use hierarchical softmax and negative sampling to speed up training.

CW considers the local context of a word $w$ as $m$ words to the left and $m$ words to the right of $w$. The concatenation of embeddings of $w$ and all its context words are taken as input of a neural network with one hidden layer, which produces a higher level representation $f(w) \in R^n$. Then the learning procedure replaces the embedding of $w$ with that of a randomly sampled word $w'$ and generate another representation $f(w') \in R^n$ with the same neural network. The training objective is to maximise the difference between them.
\begin{displaymath}
J(w, \text{ctx}(w)) = \max (0, 1 - f(w) + f(w'))
\end{displaymath}
This approach can be regarded as negative sampling with only one negative example.

Glove assumes the dot product of two word embeddings should be similar to logarithm of the co-occurrence count $X_{ij}$ of the two words. The local factor $J(w, \text{ctx}(w))$ becomes
\begin{displaymath}
g(X_{ij}) (\mathbf{v}_i^{\text{T}} \mathbf{v}_j + b_i + b_j - \log(X_{ij}))^2
\end{displaymath}
where $b_i$ and $b_j$ are the bias terms of word $i$ and $j$, $g(X_{ij})$ is a weighting function based on the co-occurence count. This weighting function controls the degree of agreement between the parametric function $\mathbf{v}_i^{\text{T}} \mathbf{v}_j + b_i + b_j $ and $\log(X_{ij})$. Frequently co-occurred word pairs will gain more weights than infrequent ones but stays the same if it is beyond a threshold.

\textit{Brown clustering} introduces a finite set of word classes $V$ for all words and partitions all words into these classes. The conditional probability of seeing the next word is defined as
\begin{displaymath}
p(w_k | w_{k - m}^{k -1}) = p(w_k | h_k) p(h_k | h_{k - m}^{k -1})
\end{displaymath}
where $h_k$ denotes the word class of the word $w_k$, $w_{k - m}^{k -1}$ are the previous $m$ words and $h_{k - m}^{k -1}$ are their respective word classes. Then $J(w, \text{ctx}(w)) = - \log p(w_k | w_{k - m}^{k -1}) $. Since there is no practical method to find optimal partition of word classes, they consider only bigram class model, and utilise hierarchical clustering as an approximation method to find a sufficiently good partition of words. 

\subsection{Building Word Representations}
\label{buildingWordRep}
For a fair comparison, we train the best performing methods Brown clustering, CBOW, Skip-gram, and Glove on a combination of freely available corpora in Table \ref{wordEmbedCorpora}. The joint corpus was preprocessed with the Stanford sentence splitter and tokenizer. All consecutive digit substrings were replaced by NUM\textit{f}, where \textit{f} is the length of the digit substring (e.g., ``10.20'' is replaced by ``NUM2.NUM2''. The word embeddings of CW are downloaded from the website \url{http://metaoptimize.com/projects/wordreprs}.

\begin{table}[h]
\begin{center}
\begin{small}
\begin{tabular}{lll}
\hline
\textbf{Data set} & \textbf{Size} & \textbf{Words} \\ \hline
UMBC 	& 48.1GB & 3 billions \\
One Billion 	& 4.1GB & 0.8 billions  \\
Latest wikipedia dump & 49.6GB & 3 billions \\ \hline
\end{tabular}
\end{small}
\caption{Corpora used to learn word embeddings}
\label{wordEmbedCorpora}
\end{center}
\end{table}

Dimension of word embedding and context window size are the key hyperparameters of the learning methods for distributed representation. We use all possible combinations from the following ranges to train word embeddings on the combined corpus.
\begin{small}
\begin{itemize}
\item[-]\textbf{Word dimension}: [25, 50, 100, 200].
\item[-]\textbf{Context window size}: [5, 10, 15].
\end{itemize}
\end{small}
Brown clustering requires only the number of clusters as hyperparameter. Thus we train word clusters with 250, 500, 1000, 2000, and 4000 clusters respectively. 
%
%
%Based on the idea that a word is characterized by the company it keeps \cite{firth1957}, 
%distributed word representation methods represent a given word as a continuous vector, which consists of the most frequent contexts of that given word in a big corpus.
%Therefore, similar words have a similar vector representation.
%
%Traditionally, the estimation of the vectors is done by initializing the vectors with co-occurrences counts. More recently, the vectors are learned in a supervised way, where
%the weights in the vectors are set to maximize the probability of the context in which the word is observed in the corpus. 
%Note that the method does no require an annotated corpus since the contexts windows used for training are extracted from unannotated data.
%
%More formally, learning a word vector $W: words \rightarrow \mathbb{R}^n$ is parametrized function that maps
%words to high-dimensional vectors, where $W$ is initialized to have random vectors for each word and learn to have meaningful vectors to perform same task.
%
%Learning the word vectors can be carry out with different models architectures. 
%In this paper, we evaluated five different word embeddings learning algorithms, which are the following: 
% 
%\begin{itemize}
%\item[-] Brown cluster \cite{Brown92class-basedn-gram}
%\item[-] CBOW \cite{Mikolov13NIPS}
%\item[-] Glove \cite{pennington2014glove}
%\item[-] Neural language model \cite{collobert2011natural}
%\item[-] Skip-Gram \cite{Mikolov13}
%\end{itemize}
%
%The above methods were chosen because they are recent
%state-of-the-art word embedding learning methods and because their software is
%available and all of them are neural networks architectures.
%The first method (Brown clusters) was selected as a benchmark word representation, which makes use of hard word clusters rather than a distributed representation.
%
%Skip-Gram \cite{Mikolov13} is a neuronal network language model, but it does not have a hidden layer, and 
%instead predicting the target word, it predicts the context given the target word.
%These embeddings are faster to train than other neuronal embeddings.
%(and does not involve dense matrix multiplication).
%
%The CBOW \cite{Mikolov13NIPS} model learns to predict the word in the middle of a symmetric window based on the sum of the vector representations of the words in the window.
%
%GloVe \cite{pennington2014glove} is essentially a log-bilinear model with a weighted least-squares objective. The main intuition underlying the model is the simple observation that ratios of word-word co-occurrence probabilities have the potential for encoding some form of meaning.
%
%%One of the benefits of neuronal networks are able to learn ways of representing the data automatically,




\section{Sequence Tagging Tasks}
\label{sec:SeqTagging}
We evaluate different word representations in four different sequence tagging tasks: POS-tagging, chunking, NER and MWE identification. 

For each sequence tagging task, we fed features into a first order linear-chain graph
transformer~\cite{collobert2011natural}, which contains two layers, as illustrated in \figref{fig:graph_transformer}. The upper layer is identical to a linear-chain CRF~\cite{lafferty2001conditional}, the lower layer consists of word representation and hand-crafted features. If we treat word representations as fixed, the graph transformer is a linear-chain CRF. Alternatively, we can treat the word representations as model parameters, in which case the model is a neural network with word embeddings as the input layer. We trained all models with the online learning algorithm AdaGrad~\cite{duchi2011adaptive}. 


\begin{figure}[t]
  \centering
  \includegraphics[scale = 0.3]{images/graph_transformer.png}
  \caption{Architecture of linear-chain graph transformer}
  \label{fig:graph_transformer}
\end{figure}


As in~\cite{turian2010word}, at each word position, we extract word representation features from the words in a window $\{-2, -1, 0, 1, 2\}$. The word representation features of word embeddings are the pre-trained continuous vectors of each word.  The Brown clustering features are the prefix features extracted from word clusters in the same way as~\cite{turian2010word}. We include the one-hot representation of unigrams for comparison as well.

For POS-tagging, chunking and MWE identification, we used the same hand-crafted features as the state-of-the-art approaches,  \newcite{collobert2011natural}, \newcite{turian2010word} and ~\newcite{mwecorpus}, respectively. For NER, we considered the same feature space as in~\cite{turian2010word}, except for the previous two predictions, because we want to evaluate all word representations with the same type of model -- a first-order graph transformer.

In training the distributed word representations, we consider two settings: 

\begin{small}
\begin{itemize}
\item[-] Word representations are fixed during sequence model training. 
\item[-] Graph transformer fine tunes the word representations during training. 
\end{itemize}
\end{small}

\begin{table*}
\begin{small}
\begin{tabular}{cccccc}
\hline
			& \textbf{Training set} & \textbf{Validation set} & \textbf{\textit{In-domain} Test set} & \textbf{\textit{Out-of-domain} Test set} \\ \hline
\textbf{POS-Tagging} & 0-18 WSJ & 19-21 WSJ & 22-24 of WSJ & English Web-Treebank  \\
\textbf{Chunking} & WSJ & 1000 sentences WSJ & CoNLL-2000 & Brown corpus \\
\textbf{NER} & CoNLL-2003 train set & CoNLL-2003 dev set & CoNLL-2003 test set & MUC7  \\
\textbf{MWE} & 500 documents from & 100 documents from & 123 documents & --- \\
\hline
\end{tabular}
\caption{Datasets splits and feature space for each sequence tagging task.}
\label{datasplit}
\end{small}
\end{table*}

We split the task-specific corpus into a training set, a validation set, and a test set (see \tabref{datasplit}). If a corpus already provides fixed splits, we reuse them. For POS-tagging,  NER and chunking, we also evaluated the models with an out-of-domain corpus (English Web Treebank, MUC-7 and Brown corpus, respectively), which have similar annotations.

For fair and reproducible experimental results, we tuned the hyperparameters with random search~\cite{bergstra2012random}. 
We randomly sampled 50 distinct hyperparameter sets with the same random seed for the models that do not update word embeddings, and sampled 100 distinct hyperparameter sets for the models that do update word embeddings. 
For each set of hyperparameters, we train a model on its training set and choose the best one based on its performance on validation data~\cite{turian2010word}. 
We also tune the word representation hyperparameters---namely, the word vector size and context window size (distributed representations) and the number of Brown clusters.
\nss{[Don't understand this sentence:]}This is achieved by mapping each possible hyperparameter combination to the word representation files trained with these parameters. 

\nss{[This paragraph needs work---I don't understand it:]}However, for the models that update word representations, we always found under-performed hyperparameters after trying out all hyperparameter combinations, because they have more hyperparameters than the models that do not update word representations. Then, for each distributed word representations, we reuse all hyperparameters of the models that do not update word representations, only tune the hyperparameters of AdaGrad for the word representation layer. This method requires only 32 additional runs for each model updating embeddings and achieves consistently better results than 100 random draws.

The final evaluation is carried out in a semi-supervised setting. We split the training set into 10 partitions at log scale (i.e., the second smallest partition will be twice the size of the smallest partition). We created 10 successively larger training sets by merging these partitions from the smallest one to the largest one, and \nss{(missing a verb?)} each of them on the same designated test sets. 

For easy comparison with previous results, we adopt the most widely used F1 measure as the evaluation metric for all tasks except POS-tagging, for which we use per-word accuracy. We also report model performance on out-of-vocabulary (unknown) words, i.e., the words that do not occur in the training set.

%\subsection{POS-tagging} We could choose one of the options. \subsubsection{Option 1} Almost the same setting as~\cite{collobert2011natural}, except adding one more test set.
% \noindent Training set: 0-18 of WSJ.
% \noindent Validation set: 19-21 of WSJ.
% \noindent Test set: 22-24 of WSJ, and English Web Treebank. We report model performances on these two test sets respectively.
% \noindent Feature space: the same set as in~\cite{collobert2011natural}

% \subsection{Chunking} The same setting as~\cite{turian2010word}\\
% \noindent Training set: WSJ train set.
% \noindent Validation set: Randomly sampled 1000 sentences from the train set for development.
% \noindent Test set: CoNLL2000 test set.
% \noindent Feature space: the same set as in~\cite{turian2010word}

% \subsection{MWE Identification} Training set: randomly sampled 500 documents from Nathana��s corpus. 
% \noindent Validation set: randomly sampled 100 documents from Nathana��s corpus.
% \noindent Test set: remaining 123 documents from Nathana��s corpus..
% \noindent Feature space: the same set as in~\cite{mwecorpus}

%\subsection{Named entity recognition} Training set: CoNLL03 train set.
% \noindent Validation set: CoNLL03 development set.
% \noindent Test set: CoNLL03 test set and MUC7. We report model performances on these two test sets respectively.
% \noindent Feature space: the same set as in~\cite{turian2010word}


%%% Local Variables: 
%%% mode: latex
%%% TeX-PDF-mode: t 
%%% TeX-master: "WordEmbEvaluation"
%%% End: 



\section{Experimental Results and Discussion}

Because we can either update pre-trained word embeddings during training or not, through the evaluation, we want to answer the following questions:
\begin{itemize}
\item How well do different word embeddings perform in all tasks when supervised fine-tuning is \textit{not} performed?
\item How well do different word embeddings perform in all tasks when supervised fine-tuning is performed?
\item How does the size of labeled training data affect the experimental results?
\item How well do the word embeddings perform for unknown words? 
\item How do the key parameters of each word learning algorithms affect the experimental results?
\end{itemize}


\begin{table*}
\caption{Benchmark results vs. our best results}
\begin{center}
\begin{small}
\begin{tabular}{lll}
\hline
\textbf{Task} & \textbf{Benchmark} & \textbf{Us} \\ \hline
POS-Tagging & (Accuracy) 97.24 \cite{Toutanova:2003} & 0.9592 (skip-gram negsam+up) \\ 
Chunking & (F1) 0.9429 \cite{Sha:2003} & 0.9386 (Brown cluster v2000+)\\  
NER & (F1) 0.8931 \cite{Ando:2005} & 0.8686 (skip-gram negsam+noup)\\  
MWE & (F1) 0.6253 \cite{Schneider+:2014} & 0.6546 (cw+up)\\ 
\hline
\label{benchmark}
\end{tabular}
\end{small}
\end{center}
\end{table*}


%%%%%%%%%%%%%%%%%%%%%%%%%%%%
%%% BEST
\begin{figure*}[h]
\caption{Best results for each method for POS-Tagging and Chunking}
\centering
\begin{subfigure}{.5\textwidth}
	\centering
    \includegraphics[width=0.8\textwidth]{plots/bestPOS}    	
	\label{fig:bestpos}
	\subcaption{POS-Tagging results}	
\end{subfigure}
\begin{subfigure}{.5\textwidth}
	\centering
    \includegraphics[width=0.8\textwidth]{plots/bestChunking}
	\label{fig:bestchunking}
	\subcaption{Chunking results}	
\end{subfigure}
\end{figure*}

\begin{figure*}[h]
\caption{Best results for each method for NER and MWE}
\centering
\begin{subfigure}{.5\textwidth}
	\centering
    	\includegraphics[width=0.8\textwidth]{plots/bestNER}
	\subcaption{NER results}	
	\label{fig:bestner}
\end{subfigure}
\begin{subfigure}{.5\textwidth}
	\centering
    	\includegraphics[width=0.8\textwidth]{plots/bestMWE}
    \subcaption{MWE results}	
	\label{fig:bestmwe}
\end{subfigure}  	
\end{figure*}  	


%%%%%%%%%%%%%%%%%%%%%%%%%%%%
%%% OOV POS
\begin{figure*}[h]
\caption{POS-Tagging out-of-vocabulary-words accuracy for \textit{in-domain} and \textit{out-of-domain} test sets}
\centering
\begin{subfigure}{.5\textwidth}
	\centering
    	\includegraphics[width=0.8\textwidth]{plots/POS-OOV-IN}
    	\subcaption{\textit{in domain} }
	\label{fig:inpos}
\end{subfigure}
\begin{subfigure}{.5\textwidth}
	\centering
    	\includegraphics[width=0.8\textwidth]{plots/POS-OOV-OUT}
   	\subcaption{\textit{out-of-domain}}
	\label{fig:outpos}
\end{subfigure}  	
\end{figure*} 

%%%%%%%%%%%%%%%%%%%%%%%%%%%%
%%% OOV NER
\begin{figure*}[h]
\caption{NER out-of-vocabulary-words accuracy for \textit{in-domain} and \textit{out-of-domain} test sets}
\centering
\begin{subfigure}{.5\textwidth}
	\centering
    	\includegraphics[width=0.8\textwidth]{plots/NER-OOV-IN}
    	\subcaption{\textit{in-domain}}
	\label{fig:inner}
\end{subfigure}
\begin{subfigure}{.5\textwidth}
	\centering
    	\includegraphics[width=0.8\textwidth]{plots/NER-OOV-OUT}
   	\subcaption{\textit{out-of-domain}}
	\label{fig:outner}
\end{subfigure}  	
\end{figure*}


%%%%%%%%%%%%%%%%%%%%%%%%%%%%
%%% OOV Chunking
\begin{figure}[h]
\caption{Chunking out-of-vocabulary-words accuracy for \textit{in-domain} test set}
\centering
    	\includegraphics[width=0.5\textwidth]{plots/Chunking-OOV}    
\label{fig:outchunking}
\end{figure}

%%%%%%%%%%%%%%%%%%%%%%%%%%%%
%%% OOV MWE
\begin{figure}[h]
\caption{MWE out-of-vocabulary-words accuracy for \textit{in-domain} test set}
\centering
    	\includegraphics[width=0.5\textwidth]{plots/MWE-OOV}    
\label{fig:outmwe}
\end{figure}








\subsection{Result tables}

The first column of each table contains the number of training sentences.

\subsubsection{Best hyperparameters for All Tasks}

\begin{table*}[h]
\centering
\begin{adjustbox}{max width=\textwidth}
\pgfplotstabletypeset[col sep=comma, 
precision=4,
every head row/.style={
before row=\toprule,
after row=\midrule},
every last row/.style={
after row=\bottomrule
}]{eval_results/key_results/POS/Accuracy.csv}
\end{adjustbox}
\caption{Accuracy of POS tagging evaluated on WSJ test set}
\label{table:accuracy_pos}
\end{table*}

\begin{table*}[h]
\centering
\begin{adjustbox}{max width=\textwidth}
\pgfplotstabletypeset[col sep=comma, 
precision=4,
every head row/.style={
before row=\toprule,
after row=\midrule},
every last row/.style={
after row=\bottomrule
}]{eval_results/key_results/NER/CONLL_F1Measure.csv}
\end{adjustbox}
\caption{F1 Measure of NER evaluated on CoNLL test set}
\label{table:f1_ner}
\end{table*}

\begin{table*}[h]
\centering
\begin{adjustbox}{max width=\textwidth}
\pgfplotstabletypeset[col sep=comma, 
precision=4,
every head row/.style={
before row=\toprule,
after row=\midrule},
every last row/.style={
after row=\bottomrule
}]{eval_results/key_results/chunking/chunks_F1Measure.csv}
\end{adjustbox}
\caption{F1 Measure of chunking evaluated on CoNLL test set}
\label{table:f1_chunking}
\end{table*}

\begin{table*}[h]
\centering
\begin{adjustbox}{max width=\textwidth}
\pgfplotstabletypeset[col sep=comma, 
precision=4,
every head row/.style={
before row=\toprule,
after row=\midrule},
every last row/.style={
after row=\bottomrule
}]{eval_results/key_results/MWEs/mwe_F1Measure.csv}
\end{adjustbox}
\caption{F1 Measure of MWE identification evaluated on MWE test set}
\label{table:f1_mwe}
\end{table*}

\subsubsection{Out of Vocabulary results for All Tasks}

\begin{table*}[h]
\centering
\begin{adjustbox}{max width=\textwidth}
\pgfplotstabletypeset[col sep=comma, 
precision=4,
every head row/.style={
before row=\toprule,
after row=\midrule},
every last row/.style={
after row=\bottomrule
}]{eval_results/key_results/POS/WSJ_out-of-vocabulary_Accuracy.csv}
\end{adjustbox}
\caption{Accuracy of POS tagging evaluated on out-of-vocabulary words in WSJ test set}
\label{table:outVocab_pos_accuracy}
\end{table*}

\begin{table*}[h]
\centering
\begin{adjustbox}{max width=\textwidth}
\pgfplotstabletypeset[col sep=comma, 
precision=4,
every head row/.style={
before row=\toprule,
after row=\midrule},
every last row/.style={
after row=\bottomrule
}]{eval_results/key_results/NER/CONLL_out-of-vocabulary_Accuracy.csv}
\end{adjustbox}
\caption{Accuracy of NER evaluated on out-of-vocabulary words in CoNLL test set}
\label{table:outVocab_ner_accuracy}
\end{table*}

\begin{table*}[h]
\centering
\begin{adjustbox}{max width=\textwidth}
\pgfplotstabletypeset[col sep=comma, 
precision=4,
every head row/.style={
before row=\toprule,
after row=\midrule},
every last row/.style={
after row=\bottomrule
}]{eval_results/key_results/chunking/Accuracy.csv}
\end{adjustbox}
\caption{Accuracy of Chunking evaluated on out-of-vocabulary words in CoNLL test set}
\label{table:outVocab_chunking_accuracy}
\end{table*}

\begin{table*}[h]
\centering
\begin{adjustbox}{max width=\textwidth}
\pgfplotstabletypeset[col sep=comma, 
precision=4,
every head row/.style={
before row=\toprule,
after row=\midrule},
every last row/.style={
after row=\bottomrule
}]{eval_results/key_results/MWEs/out-of-vocabulary_Accuracy.csv}
\end{adjustbox}
\caption{Accuracy of MWE identification evaluated on out-of-vocabulary words in MWE test set}
\label{table:outVocab_mwe_accuracy}
\end{table*}

\subsubsection{Out of domain Results for NER and POS}

\begin{table*}[h]
\centering
\begin{adjustbox}{max width=\textwidth}
\pgfplotstabletypeset[col sep=comma, 
precision=4,
every head row/.style={
before row=\toprule,
after row=\midrule},
every last row/.style={
after row=\bottomrule
}]{eval_results/key_results/POS/EngWebTreebank_Accuracy.csv}
\end{adjustbox}
\caption{Accuracy of POS tagging evaluated on English web treebank.}
\label{table:outDomain_accuracy_pos}
\end{table*}

\begin{table*}[h]
\centering
\begin{adjustbox}{max width=\textwidth}
\pgfplotstabletypeset[
col sep=comma, 
precision=4,
every head row/.style={
before row=\toprule,
after row=\midrule},
every last row/.style={
after row=\bottomrule
}]{eval_results/key_results/NER/MUC7_chunks_F1Measure.csv}
\end{adjustbox}
\caption{F1 measure of NER evaluated on MUC7 test set}
\label{table:outDomain_ner_f1}
\end{table*}


\section{Conclusion}

\section{Related Work}
%Word embedding learning methods have been applied to several 
%NLP tasks that we summaries in this section.

\newcite{collobert2011natural} proposed a neural network architecture
that learns word embeddings and uses them in POS-tagging, chunking, NER and semantic role labelling. 
Without specializing their architecture for the task, they achieve close to state-of-the-art performance. After including specialized features (e.g., word suffixes for POS-tagging;  gazetteers for NER, etc.) and other tricks like cascading and ensembling classifiers, they achieve competitive state-of-the-art performance.
% they system is very fast too.
Similarly, \newcite{turian2010word} explored the impact for NER and chunking 
of %using word
features derived from word clusters and embeddings. 
They conclude that unsupervised word representations improve NER and chunking, and that combining different word representations can further improve the performance.
Brown clusters have been also shown to enhance
Twitter POS tagging \newcite{owoputi2013improved}. 

\newcite{Schneider+:2014} presented an MWE analyser that, among other features, used Brown clusters. 
They observed that the clusters were useful for identifying words that usually belong to multiword proper names, which are considered MWEs in the dataset used. Nevertheless, they mentioned that it is difficult to ascertain the impact of the word embedding features, since other features may capture the same information. 
%Further feature engineering should be carry out in order to find out the impact 

Word embeddings have been also used as features for syntactic dependency parsing and constituency parsing. 
\newcite{Bansal+:2014} used word embeddings as features for dependency parsing, which used the syntactic dependency context instead of the linear context in raw text. They found that simple attempts based on discretization of individual word vector dimensions do not improve parsing. Only when performing hierarchical clustering of the continuous word vectors, then using features based on the hierarchy, do they see an improvement. They also found that an ensemble of different word embeddings improved performance.
In a similar vein, \newcite{Andreas:Klein:2014} explored the used of word embeddings for constituency parsing and concluded that the
information they provide might be redundant with that acquired by a syntactic parser trained with a small amount of data. Other syntactic parsing studies reporting improvements from word embeddings include \cite{Koo:2008,Koo:2010,Haffari:2011,Tratz:2011}.

Word embeddings have also been applied to other (non-sequential NLP) tasks like grammar induction \cite{Spitkovsky:2011} and semantic tasks such as semantic relatedness, synonymy detection, concept categorization, selectional preferences and analogy \cite{baroni:2014}.

\nss{[Guo et al. 2014?]}

% such as super-sense tagging \cite{Grave:2013};

% Koo, T., Carreras, X., & Collins, M. (2008). Simple semi-supervised dependency parsing. ACL (pp. 595?603).
% Koo et al., 2008; http://cs.nyu.edu/~dsontag/papers/KooEtAl_emnlp10.pdf Dual Decomposition for Parsing with Non-Projective Head Automata - Dependency parsing.

%Haffari et al., 2011; http://www.aclweb.org/anthology/P11-2125 Ensemble of different dependency parsing models, each model corresponding to a different syntactic/semantic word clustering annotation.

% Supersense tagger (Grave et al., 2013) https://hal.inria.fr/hal-00833288/PDF/final-version.pdf


\section{Conclusions}

We have performed an extensive extrinsic evaluation of five word embedding methods
under fixed experimental conditions, and evaluated their applicability to four sequence tagging tasks: \pos, \chunking, \ner and \mwe identification.
We found that word embedding features reliably outperformed unigram
features, especially with limited training data, but that there was
relatively little difference over Brown clusters, and
no one embedding method was consistently superior across the different tasks and settings.
Word embeddings and Brown clusters were also found to improve
out-of-domain performance and for OOV words.
We expected a performance gap between the fixed and task-updated embeddings, but the observed difference was marginal.
Indeed, we found that updating can result in overfitting.
We also carried out preliminary analysis of the impact of updating on
the vectors, a direction which we intend to pursue further.

% Finally, by using word embeddings as features for MWE identification, we outperformed 
% the state-of-the-art system on that task.
% %We could not find any trend that suggest that a word embedding method is better than other.
% We hope to build on this in future work by learning representations directly over multiword units.\nss{[Has there been any research into this issue?]}


%%% Local Variables: 
%%% mode: latex
%%% TeX-PDF-mode: t 
%%% TeX-master: "WordEmbEvaluation"
%%% End: 


% \section*{Acknowledgments}

% Anonymised\\
% Anonymised\\
% Anonymised\\
% Anonymised\\
% Anonymised\\
% Anonymised\\

%NICTA is funded by the Australian Government as represented by the Department of Broadband, Communications and the Digital Economy and the Australian Research Council through the ICT Centre of Excellence program.

\bibliographystyle{acl2013}
\bibliography{biblio}

\end{document}


%%% Local Variables: 
%%% mode: latex
%%% TeX-PDF-mode: t 
%%% TeX-master: t
%%% End: 
